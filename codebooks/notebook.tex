
% Default to the notebook output style

    


% Inherit from the specified cell style.




    
\documentclass[11pt]{article}

    
    
    \usepackage[T1]{fontenc}
    % Nicer default font (+ math font) than Computer Modern for most use cases
    \usepackage{mathpazo}

    % Basic figure setup, for now with no caption control since it's done
    % automatically by Pandoc (which extracts ![](path) syntax from Markdown).
    \usepackage{graphicx}
    % We will generate all images so they have a width \maxwidth. This means
    % that they will get their normal width if they fit onto the page, but
    % are scaled down if they would overflow the margins.
    \makeatletter
    \def\maxwidth{\ifdim\Gin@nat@width>\linewidth\linewidth
    \else\Gin@nat@width\fi}
    \makeatother
    \let\Oldincludegraphics\includegraphics
    % Set max figure width to be 80% of text width, for now hardcoded.
    \renewcommand{\includegraphics}[1]{\Oldincludegraphics[width=.8\maxwidth]{#1}}
    % Ensure that by default, figures have no caption (until we provide a
    % proper Figure object with a Caption API and a way to capture that
    % in the conversion process - todo).
    \usepackage{caption}
    \DeclareCaptionLabelFormat{nolabel}{}
    \captionsetup{labelformat=nolabel}

    \usepackage{adjustbox} % Used to constrain images to a maximum size 
    \usepackage{xcolor} % Allow colors to be defined
    \usepackage{enumerate} % Needed for markdown enumerations to work
    \usepackage{geometry} % Used to adjust the document margins
    \usepackage{amsmath} % Equations
    \usepackage{amssymb} % Equations
    \usepackage{textcomp} % defines textquotesingle
    % Hack from http://tex.stackexchange.com/a/47451/13684:
    \AtBeginDocument{%
        \def\PYZsq{\textquotesingle}% Upright quotes in Pygmentized code
    }
    \usepackage{upquote} % Upright quotes for verbatim code
    \usepackage{eurosym} % defines \euro
    \usepackage[mathletters]{ucs} % Extended unicode (utf-8) support
    \usepackage[utf8x]{inputenc} % Allow utf-8 characters in the tex document
    \usepackage{fancyvrb} % verbatim replacement that allows latex
    \usepackage{grffile} % extends the file name processing of package graphics 
                         % to support a larger range 
    % The hyperref package gives us a pdf with properly built
    % internal navigation ('pdf bookmarks' for the table of contents,
    % internal cross-reference links, web links for URLs, etc.)
    \usepackage{hyperref}
    \usepackage{longtable} % longtable support required by pandoc >1.10
    \usepackage{booktabs}  % table support for pandoc > 1.12.2
    \usepackage[inline]{enumitem} % IRkernel/repr support (it uses the enumerate* environment)
    \usepackage[normalem]{ulem} % ulem is needed to support strikethroughs (\sout)
                                % normalem makes italics be italics, not underlines
    

    
    
    % Colors for the hyperref package
    \definecolor{urlcolor}{rgb}{0,.145,.698}
    \definecolor{linkcolor}{rgb}{.71,0.21,0.01}
    \definecolor{citecolor}{rgb}{.12,.54,.11}

    % ANSI colors
    \definecolor{ansi-black}{HTML}{3E424D}
    \definecolor{ansi-black-intense}{HTML}{282C36}
    \definecolor{ansi-red}{HTML}{E75C58}
    \definecolor{ansi-red-intense}{HTML}{B22B31}
    \definecolor{ansi-green}{HTML}{00A250}
    \definecolor{ansi-green-intense}{HTML}{007427}
    \definecolor{ansi-yellow}{HTML}{DDB62B}
    \definecolor{ansi-yellow-intense}{HTML}{B27D12}
    \definecolor{ansi-blue}{HTML}{208FFB}
    \definecolor{ansi-blue-intense}{HTML}{0065CA}
    \definecolor{ansi-magenta}{HTML}{D160C4}
    \definecolor{ansi-magenta-intense}{HTML}{A03196}
    \definecolor{ansi-cyan}{HTML}{60C6C8}
    \definecolor{ansi-cyan-intense}{HTML}{258F8F}
    \definecolor{ansi-white}{HTML}{C5C1B4}
    \definecolor{ansi-white-intense}{HTML}{A1A6B2}

    % commands and environments needed by pandoc snippets
    % extracted from the output of `pandoc -s`
    \providecommand{\tightlist}{%
      \setlength{\itemsep}{0pt}\setlength{\parskip}{0pt}}
    \DefineVerbatimEnvironment{Highlighting}{Verbatim}{commandchars=\\\{\}}
    % Add ',fontsize=\small' for more characters per line
    \newenvironment{Shaded}{}{}
    \newcommand{\KeywordTok}[1]{\textcolor[rgb]{0.00,0.44,0.13}{\textbf{{#1}}}}
    \newcommand{\DataTypeTok}[1]{\textcolor[rgb]{0.56,0.13,0.00}{{#1}}}
    \newcommand{\DecValTok}[1]{\textcolor[rgb]{0.25,0.63,0.44}{{#1}}}
    \newcommand{\BaseNTok}[1]{\textcolor[rgb]{0.25,0.63,0.44}{{#1}}}
    \newcommand{\FloatTok}[1]{\textcolor[rgb]{0.25,0.63,0.44}{{#1}}}
    \newcommand{\CharTok}[1]{\textcolor[rgb]{0.25,0.44,0.63}{{#1}}}
    \newcommand{\StringTok}[1]{\textcolor[rgb]{0.25,0.44,0.63}{{#1}}}
    \newcommand{\CommentTok}[1]{\textcolor[rgb]{0.38,0.63,0.69}{\textit{{#1}}}}
    \newcommand{\OtherTok}[1]{\textcolor[rgb]{0.00,0.44,0.13}{{#1}}}
    \newcommand{\AlertTok}[1]{\textcolor[rgb]{1.00,0.00,0.00}{\textbf{{#1}}}}
    \newcommand{\FunctionTok}[1]{\textcolor[rgb]{0.02,0.16,0.49}{{#1}}}
    \newcommand{\RegionMarkerTok}[1]{{#1}}
    \newcommand{\ErrorTok}[1]{\textcolor[rgb]{1.00,0.00,0.00}{\textbf{{#1}}}}
    \newcommand{\NormalTok}[1]{{#1}}
    
    % Additional commands for more recent versions of Pandoc
    \newcommand{\ConstantTok}[1]{\textcolor[rgb]{0.53,0.00,0.00}{{#1}}}
    \newcommand{\SpecialCharTok}[1]{\textcolor[rgb]{0.25,0.44,0.63}{{#1}}}
    \newcommand{\VerbatimStringTok}[1]{\textcolor[rgb]{0.25,0.44,0.63}{{#1}}}
    \newcommand{\SpecialStringTok}[1]{\textcolor[rgb]{0.73,0.40,0.53}{{#1}}}
    \newcommand{\ImportTok}[1]{{#1}}
    \newcommand{\DocumentationTok}[1]{\textcolor[rgb]{0.73,0.13,0.13}{\textit{{#1}}}}
    \newcommand{\AnnotationTok}[1]{\textcolor[rgb]{0.38,0.63,0.69}{\textbf{\textit{{#1}}}}}
    \newcommand{\CommentVarTok}[1]{\textcolor[rgb]{0.38,0.63,0.69}{\textbf{\textit{{#1}}}}}
    \newcommand{\VariableTok}[1]{\textcolor[rgb]{0.10,0.09,0.49}{{#1}}}
    \newcommand{\ControlFlowTok}[1]{\textcolor[rgb]{0.00,0.44,0.13}{\textbf{{#1}}}}
    \newcommand{\OperatorTok}[1]{\textcolor[rgb]{0.40,0.40,0.40}{{#1}}}
    \newcommand{\BuiltInTok}[1]{{#1}}
    \newcommand{\ExtensionTok}[1]{{#1}}
    \newcommand{\PreprocessorTok}[1]{\textcolor[rgb]{0.74,0.48,0.00}{{#1}}}
    \newcommand{\AttributeTok}[1]{\textcolor[rgb]{0.49,0.56,0.16}{{#1}}}
    \newcommand{\InformationTok}[1]{\textcolor[rgb]{0.38,0.63,0.69}{\textbf{\textit{{#1}}}}}
    \newcommand{\WarningTok}[1]{\textcolor[rgb]{0.38,0.63,0.69}{\textbf{\textit{{#1}}}}}
    
    
    % Define a nice break command that doesn't care if a line doesn't already
    % exist.
    \def\br{\hspace*{\fill} \\* }
    % Math Jax compatability definitions
    \def\gt{>}
    \def\lt{<}
    % Document parameters
    \title{Clustering with satellite images}
    
    
    

    % Pygments definitions
    
\makeatletter
\def\PY@reset{\let\PY@it=\relax \let\PY@bf=\relax%
    \let\PY@ul=\relax \let\PY@tc=\relax%
    \let\PY@bc=\relax \let\PY@ff=\relax}
\def\PY@tok#1{\csname PY@tok@#1\endcsname}
\def\PY@toks#1+{\ifx\relax#1\empty\else%
    \PY@tok{#1}\expandafter\PY@toks\fi}
\def\PY@do#1{\PY@bc{\PY@tc{\PY@ul{%
    \PY@it{\PY@bf{\PY@ff{#1}}}}}}}
\def\PY#1#2{\PY@reset\PY@toks#1+\relax+\PY@do{#2}}

\expandafter\def\csname PY@tok@nv\endcsname{\def\PY@tc##1{\textcolor[rgb]{0.10,0.09,0.49}{##1}}}
\expandafter\def\csname PY@tok@gs\endcsname{\let\PY@bf=\textbf}
\expandafter\def\csname PY@tok@ow\endcsname{\let\PY@bf=\textbf\def\PY@tc##1{\textcolor[rgb]{0.67,0.13,1.00}{##1}}}
\expandafter\def\csname PY@tok@cp\endcsname{\def\PY@tc##1{\textcolor[rgb]{0.74,0.48,0.00}{##1}}}
\expandafter\def\csname PY@tok@nf\endcsname{\def\PY@tc##1{\textcolor[rgb]{0.00,0.00,1.00}{##1}}}
\expandafter\def\csname PY@tok@nn\endcsname{\let\PY@bf=\textbf\def\PY@tc##1{\textcolor[rgb]{0.00,0.00,1.00}{##1}}}
\expandafter\def\csname PY@tok@kt\endcsname{\def\PY@tc##1{\textcolor[rgb]{0.69,0.00,0.25}{##1}}}
\expandafter\def\csname PY@tok@kd\endcsname{\let\PY@bf=\textbf\def\PY@tc##1{\textcolor[rgb]{0.00,0.50,0.00}{##1}}}
\expandafter\def\csname PY@tok@w\endcsname{\def\PY@tc##1{\textcolor[rgb]{0.73,0.73,0.73}{##1}}}
\expandafter\def\csname PY@tok@c1\endcsname{\let\PY@it=\textit\def\PY@tc##1{\textcolor[rgb]{0.25,0.50,0.50}{##1}}}
\expandafter\def\csname PY@tok@s\endcsname{\def\PY@tc##1{\textcolor[rgb]{0.73,0.13,0.13}{##1}}}
\expandafter\def\csname PY@tok@gp\endcsname{\let\PY@bf=\textbf\def\PY@tc##1{\textcolor[rgb]{0.00,0.00,0.50}{##1}}}
\expandafter\def\csname PY@tok@ss\endcsname{\def\PY@tc##1{\textcolor[rgb]{0.10,0.09,0.49}{##1}}}
\expandafter\def\csname PY@tok@s2\endcsname{\def\PY@tc##1{\textcolor[rgb]{0.73,0.13,0.13}{##1}}}
\expandafter\def\csname PY@tok@go\endcsname{\def\PY@tc##1{\textcolor[rgb]{0.53,0.53,0.53}{##1}}}
\expandafter\def\csname PY@tok@nl\endcsname{\def\PY@tc##1{\textcolor[rgb]{0.63,0.63,0.00}{##1}}}
\expandafter\def\csname PY@tok@nc\endcsname{\let\PY@bf=\textbf\def\PY@tc##1{\textcolor[rgb]{0.00,0.00,1.00}{##1}}}
\expandafter\def\csname PY@tok@vc\endcsname{\def\PY@tc##1{\textcolor[rgb]{0.10,0.09,0.49}{##1}}}
\expandafter\def\csname PY@tok@gi\endcsname{\def\PY@tc##1{\textcolor[rgb]{0.00,0.63,0.00}{##1}}}
\expandafter\def\csname PY@tok@sc\endcsname{\def\PY@tc##1{\textcolor[rgb]{0.73,0.13,0.13}{##1}}}
\expandafter\def\csname PY@tok@nd\endcsname{\def\PY@tc##1{\textcolor[rgb]{0.67,0.13,1.00}{##1}}}
\expandafter\def\csname PY@tok@nb\endcsname{\def\PY@tc##1{\textcolor[rgb]{0.00,0.50,0.00}{##1}}}
\expandafter\def\csname PY@tok@bp\endcsname{\def\PY@tc##1{\textcolor[rgb]{0.00,0.50,0.00}{##1}}}
\expandafter\def\csname PY@tok@m\endcsname{\def\PY@tc##1{\textcolor[rgb]{0.40,0.40,0.40}{##1}}}
\expandafter\def\csname PY@tok@na\endcsname{\def\PY@tc##1{\textcolor[rgb]{0.49,0.56,0.16}{##1}}}
\expandafter\def\csname PY@tok@c\endcsname{\let\PY@it=\textit\def\PY@tc##1{\textcolor[rgb]{0.25,0.50,0.50}{##1}}}
\expandafter\def\csname PY@tok@sr\endcsname{\def\PY@tc##1{\textcolor[rgb]{0.73,0.40,0.53}{##1}}}
\expandafter\def\csname PY@tok@ni\endcsname{\let\PY@bf=\textbf\def\PY@tc##1{\textcolor[rgb]{0.60,0.60,0.60}{##1}}}
\expandafter\def\csname PY@tok@kn\endcsname{\let\PY@bf=\textbf\def\PY@tc##1{\textcolor[rgb]{0.00,0.50,0.00}{##1}}}
\expandafter\def\csname PY@tok@sd\endcsname{\let\PY@it=\textit\def\PY@tc##1{\textcolor[rgb]{0.73,0.13,0.13}{##1}}}
\expandafter\def\csname PY@tok@sx\endcsname{\def\PY@tc##1{\textcolor[rgb]{0.00,0.50,0.00}{##1}}}
\expandafter\def\csname PY@tok@vm\endcsname{\def\PY@tc##1{\textcolor[rgb]{0.10,0.09,0.49}{##1}}}
\expandafter\def\csname PY@tok@nt\endcsname{\let\PY@bf=\textbf\def\PY@tc##1{\textcolor[rgb]{0.00,0.50,0.00}{##1}}}
\expandafter\def\csname PY@tok@gd\endcsname{\def\PY@tc##1{\textcolor[rgb]{0.63,0.00,0.00}{##1}}}
\expandafter\def\csname PY@tok@k\endcsname{\let\PY@bf=\textbf\def\PY@tc##1{\textcolor[rgb]{0.00,0.50,0.00}{##1}}}
\expandafter\def\csname PY@tok@mf\endcsname{\def\PY@tc##1{\textcolor[rgb]{0.40,0.40,0.40}{##1}}}
\expandafter\def\csname PY@tok@gt\endcsname{\def\PY@tc##1{\textcolor[rgb]{0.00,0.27,0.87}{##1}}}
\expandafter\def\csname PY@tok@ne\endcsname{\let\PY@bf=\textbf\def\PY@tc##1{\textcolor[rgb]{0.82,0.25,0.23}{##1}}}
\expandafter\def\csname PY@tok@se\endcsname{\let\PY@bf=\textbf\def\PY@tc##1{\textcolor[rgb]{0.73,0.40,0.13}{##1}}}
\expandafter\def\csname PY@tok@vg\endcsname{\def\PY@tc##1{\textcolor[rgb]{0.10,0.09,0.49}{##1}}}
\expandafter\def\csname PY@tok@dl\endcsname{\def\PY@tc##1{\textcolor[rgb]{0.73,0.13,0.13}{##1}}}
\expandafter\def\csname PY@tok@cpf\endcsname{\let\PY@it=\textit\def\PY@tc##1{\textcolor[rgb]{0.25,0.50,0.50}{##1}}}
\expandafter\def\csname PY@tok@kp\endcsname{\def\PY@tc##1{\textcolor[rgb]{0.00,0.50,0.00}{##1}}}
\expandafter\def\csname PY@tok@gr\endcsname{\def\PY@tc##1{\textcolor[rgb]{1.00,0.00,0.00}{##1}}}
\expandafter\def\csname PY@tok@sb\endcsname{\def\PY@tc##1{\textcolor[rgb]{0.73,0.13,0.13}{##1}}}
\expandafter\def\csname PY@tok@gu\endcsname{\let\PY@bf=\textbf\def\PY@tc##1{\textcolor[rgb]{0.50,0.00,0.50}{##1}}}
\expandafter\def\csname PY@tok@cs\endcsname{\let\PY@it=\textit\def\PY@tc##1{\textcolor[rgb]{0.25,0.50,0.50}{##1}}}
\expandafter\def\csname PY@tok@fm\endcsname{\def\PY@tc##1{\textcolor[rgb]{0.00,0.00,1.00}{##1}}}
\expandafter\def\csname PY@tok@mh\endcsname{\def\PY@tc##1{\textcolor[rgb]{0.40,0.40,0.40}{##1}}}
\expandafter\def\csname PY@tok@ch\endcsname{\let\PY@it=\textit\def\PY@tc##1{\textcolor[rgb]{0.25,0.50,0.50}{##1}}}
\expandafter\def\csname PY@tok@vi\endcsname{\def\PY@tc##1{\textcolor[rgb]{0.10,0.09,0.49}{##1}}}
\expandafter\def\csname PY@tok@kr\endcsname{\let\PY@bf=\textbf\def\PY@tc##1{\textcolor[rgb]{0.00,0.50,0.00}{##1}}}
\expandafter\def\csname PY@tok@sh\endcsname{\def\PY@tc##1{\textcolor[rgb]{0.73,0.13,0.13}{##1}}}
\expandafter\def\csname PY@tok@s1\endcsname{\def\PY@tc##1{\textcolor[rgb]{0.73,0.13,0.13}{##1}}}
\expandafter\def\csname PY@tok@il\endcsname{\def\PY@tc##1{\textcolor[rgb]{0.40,0.40,0.40}{##1}}}
\expandafter\def\csname PY@tok@mb\endcsname{\def\PY@tc##1{\textcolor[rgb]{0.40,0.40,0.40}{##1}}}
\expandafter\def\csname PY@tok@kc\endcsname{\let\PY@bf=\textbf\def\PY@tc##1{\textcolor[rgb]{0.00,0.50,0.00}{##1}}}
\expandafter\def\csname PY@tok@gh\endcsname{\let\PY@bf=\textbf\def\PY@tc##1{\textcolor[rgb]{0.00,0.00,0.50}{##1}}}
\expandafter\def\csname PY@tok@mo\endcsname{\def\PY@tc##1{\textcolor[rgb]{0.40,0.40,0.40}{##1}}}
\expandafter\def\csname PY@tok@o\endcsname{\def\PY@tc##1{\textcolor[rgb]{0.40,0.40,0.40}{##1}}}
\expandafter\def\csname PY@tok@err\endcsname{\def\PY@bc##1{\setlength{\fboxsep}{0pt}\fcolorbox[rgb]{1.00,0.00,0.00}{1,1,1}{\strut ##1}}}
\expandafter\def\csname PY@tok@no\endcsname{\def\PY@tc##1{\textcolor[rgb]{0.53,0.00,0.00}{##1}}}
\expandafter\def\csname PY@tok@ge\endcsname{\let\PY@it=\textit}
\expandafter\def\csname PY@tok@mi\endcsname{\def\PY@tc##1{\textcolor[rgb]{0.40,0.40,0.40}{##1}}}
\expandafter\def\csname PY@tok@cm\endcsname{\let\PY@it=\textit\def\PY@tc##1{\textcolor[rgb]{0.25,0.50,0.50}{##1}}}
\expandafter\def\csname PY@tok@si\endcsname{\let\PY@bf=\textbf\def\PY@tc##1{\textcolor[rgb]{0.73,0.40,0.53}{##1}}}
\expandafter\def\csname PY@tok@sa\endcsname{\def\PY@tc##1{\textcolor[rgb]{0.73,0.13,0.13}{##1}}}

\def\PYZbs{\char`\\}
\def\PYZus{\char`\_}
\def\PYZob{\char`\{}
\def\PYZcb{\char`\}}
\def\PYZca{\char`\^}
\def\PYZam{\char`\&}
\def\PYZlt{\char`\<}
\def\PYZgt{\char`\>}
\def\PYZsh{\char`\#}
\def\PYZpc{\char`\%}
\def\PYZdl{\char`\$}
\def\PYZhy{\char`\-}
\def\PYZsq{\char`\'}
\def\PYZdq{\char`\"}
\def\PYZti{\char`\~}
% for compatibility with earlier versions
\def\PYZat{@}
\def\PYZlb{[}
\def\PYZrb{]}
\makeatother


    % Exact colors from NB
    \definecolor{incolor}{rgb}{0.0, 0.0, 0.5}
    \definecolor{outcolor}{rgb}{0.545, 0.0, 0.0}



    
    % Prevent overflowing lines due to hard-to-break entities
    \sloppy 
    % Setup hyperref package
    \hypersetup{
      breaklinks=true,  % so long urls are correctly broken across lines
      colorlinks=true,
      urlcolor=urlcolor,
      linkcolor=linkcolor,
      citecolor=citecolor,
      }
    % Slightly bigger margins than the latex defaults
    
    \geometry{verbose,tmargin=1in,bmargin=1in,lmargin=1in,rmargin=1in}
    
    

    \begin{document}
    
    
    \maketitle
    
    

    
    \section{Applying clustering to identify land use in Satellite
imagery}\label{applying-clustering-to-identify-land-use-in-satellite-imagery}

    \begin{Verbatim}[commandchars=\\\{\}]
{\color{incolor}In [{\color{incolor}256}]:} \PY{k+kn}{import} \PY{n+nn}{os}
          
          \PY{c+c1}{\PYZsh{}data manipulation}
          \PY{k+kn}{import} \PY{n+nn}{numpy} \PY{k}{as} \PY{n+nn}{np}
          
          
          \PY{c+c1}{\PYZsh{}reading and displying images}
          \PY{k+kn}{import} \PY{n+nn}{matplotlib}\PY{n+nn}{.}\PY{n+nn}{pyplot} \PY{k}{as} \PY{n+nn}{plt}
          \PY{k+kn}{import} \PY{n+nn}{seaborn} \PY{k}{as} \PY{n+nn}{sns}
          
          \PY{c+c1}{\PYZsh{}the K\PYZhy{}means implementation}
          \PY{k+kn}{from} \PY{n+nn}{sklearn}\PY{n+nn}{.}\PY{n+nn}{cluster} \PY{k}{import} \PY{n}{KMeans}
          
          \PY{c+c1}{\PYZsh{}guassaian smoothing}
          \PY{k+kn}{from} \PY{n+nn}{scipy}\PY{n+nn}{.}\PY{n+nn}{ndimage} \PY{k}{import} \PY{n}{gaussian\PYZus{}filter}
          
          \PY{c+c1}{\PYZsh{}inline plots}
          \PY{o}{\PYZpc{}}\PY{k}{matplotlib} inline
          \PY{n}{plt}\PY{o}{.}\PY{n}{rcParams}\PY{p}{[}\PY{l+s+s2}{\PYZdq{}}\PY{l+s+s2}{figure.figsize}\PY{l+s+s2}{\PYZdq{}}\PY{p}{]} \PY{o}{=} \PY{p}{(}\PY{l+m+mi}{20}\PY{p}{,}\PY{l+m+mi}{20}\PY{p}{)}
\end{Verbatim}


    \section{The Data}\label{the-data}

    Lets try some satellite images from:
https://apps.sentinel-hub.com/sentinel-playground

We load several images to try. One is the natural image; what you see
with your eyes. The others include several different spectra highlighing
argicultural, urban and vegetation.

    \begin{Verbatim}[commandchars=\\\{\}]
{\color{incolor}In [{\color{incolor}164}]:} \PY{n}{files} \PY{o}{=} \PY{p}{[}\PY{l+s+s2}{\PYZdq{}}\PY{l+s+s2}{Sentinel\PYZhy{}2 image on 2018\PYZhy{}012\PYZhy{}natural.jpg}\PY{l+s+s2}{\PYZdq{}}\PY{p}{,}
                   \PY{l+s+s2}{\PYZdq{}}\PY{l+s+s2}{Sentinel\PYZhy{}2 image on 2018\PYZhy{}01\PYZhy{}12\PYZhy{}agric.jpg}\PY{l+s+s2}{\PYZdq{}}\PY{p}{,}
                   \PY{l+s+s2}{\PYZdq{}}\PY{l+s+s2}{Sentinel\PYZhy{}2 image on 2018\PYZhy{}01\PYZhy{}12\PYZhy{}urban.jpg}\PY{l+s+s2}{\PYZdq{}}\PY{p}{,}
                   \PY{l+s+s2}{\PYZdq{}}\PY{l+s+s2}{Sentinel\PYZhy{}2 image on 2018\PYZhy{}01\PYZhy{}12\PYZhy{}vegetation.jpg}\PY{l+s+s2}{\PYZdq{}}
                  \PY{p}{]}
          
          \PY{n}{names} \PY{o}{=} \PY{p}{[}\PY{l+s+s2}{\PYZdq{}}\PY{l+s+s2}{Natural}\PY{l+s+s2}{\PYZdq{}}\PY{p}{,}
                   \PY{l+s+s2}{\PYZdq{}}\PY{l+s+s2}{Agricultural}\PY{l+s+s2}{\PYZdq{}}\PY{p}{,}
                   \PY{l+s+s2}{\PYZdq{}}\PY{l+s+s2}{Urban}\PY{l+s+s2}{\PYZdq{}}\PY{p}{,}
                   \PY{l+s+s2}{\PYZdq{}}\PY{l+s+s2}{Vegetation}\PY{l+s+s2}{\PYZdq{}}\PY{p}{]}
\end{Verbatim}


    Let's read the files into a dict

    \begin{Verbatim}[commandchars=\\\{\}]
{\color{incolor}In [{\color{incolor}211}]:} \PY{n}{file\PYZus{}dir} \PY{o}{=} \PY{l+s+s2}{\PYZdq{}}\PY{l+s+s2}{../data/ghana\PYZus{}data/}\PY{l+s+s2}{\PYZdq{}}
          
          \PY{n}{images} \PY{o}{=} \PY{p}{[}\PY{n}{plt}\PY{o}{.}\PY{n}{imread}\PY{p}{(}\PY{n}{file\PYZus{}dir} \PY{o}{+} \PY{n}{file}\PY{p}{)} \PY{k}{for} \PY{n}{file} \PY{o+ow}{in} \PY{n}{files}\PY{p}{]}
          \PY{n}{images} \PY{o}{=} \PY{n+nb}{dict}\PY{p}{(}\PY{n+nb}{zip}\PY{p}{(}\PY{n}{names}\PY{p}{,} \PY{n}{images}\PY{p}{)}\PY{p}{)}
\end{Verbatim}


    \subsection{Pre-processing the image}\label{pre-processing-the-image}

We can try to pre-process the image by applying a gaussian smoothing
function. This will mean we lose some fine detail but we are not
interested in that anyway.

This will give us some idea if pre-processing is best for the images.

    \begin{Verbatim}[commandchars=\\\{\}]
{\color{incolor}In [{\color{incolor}234}]:} \PY{n}{smooth\PYZus{}imgs} \PY{o}{=} \PY{p}{[}\PY{p}{]}
          
          \PY{k}{for} \PY{n}{name} \PY{o+ow}{in} \PY{n}{names}\PY{p}{:}
              \PY{n}{smooth\PYZus{}imgs}\PY{o}{.}\PY{n}{append}\PY{p}{(}\PY{n}{gaussian\PYZus{}filter}\PY{p}{(}\PY{n}{images}\PY{p}{[}\PY{n}{name}\PY{p}{]}\PY{p}{,} \PY{n}{sigma} \PY{o}{=} \PY{p}{[}\PY{l+m+mi}{5}\PY{p}{,}\PY{l+m+mi}{5}\PY{p}{,}\PY{l+m+mi}{0}\PY{p}{]}\PY{p}{)}\PY{p}{)}
              
          \PY{n}{smooth\PYZus{}images} \PY{o}{=} \PY{n+nb}{dict}\PY{p}{(}\PY{n+nb}{zip}\PY{p}{(}\PY{n}{names}\PY{p}{,} \PY{n}{smooth\PYZus{}imgs}\PY{p}{)}\PY{p}{)}
\end{Verbatim}


    Ok, let's look at the images

    \begin{Verbatim}[commandchars=\\\{\}]
{\color{incolor}In [{\color{incolor}259}]:} \PY{k}{for} \PY{n}{name} \PY{o+ow}{in} \PY{n}{names}\PY{p}{:}
           
              \PY{n}{fig}\PY{p}{,} \PY{n}{axs} \PY{o}{=} \PY{n}{plt}\PY{o}{.}\PY{n}{subplots}\PY{p}{(}\PY{l+m+mi}{1}\PY{p}{,}\PY{l+m+mi}{2}\PY{p}{)}
              \PY{n}{axs}\PY{p}{[}\PY{l+m+mi}{0}\PY{p}{]}\PY{o}{.}\PY{n}{imshow}\PY{p}{(}\PY{n}{images}\PY{p}{[}\PY{n}{name}\PY{p}{]}\PY{p}{)}
              \PY{n}{axs}\PY{p}{[}\PY{l+m+mi}{0}\PY{p}{]}\PY{o}{.}\PY{n}{set\PYZus{}title}\PY{p}{(}\PY{n}{name} \PY{o}{+} \PY{l+s+s2}{\PYZdq{}}\PY{l+s+s2}{: Unprocessed}\PY{l+s+s2}{\PYZdq{}}\PY{p}{)}
              \PY{n}{axs}\PY{p}{[}\PY{l+m+mi}{1}\PY{p}{]}\PY{o}{.}\PY{n}{imshow}\PY{p}{(}\PY{n}{smooth\PYZus{}images}\PY{p}{[}\PY{n}{name}\PY{p}{]}\PY{p}{)}
              \PY{n}{axs}\PY{p}{[}\PY{l+m+mi}{1}\PY{p}{]}\PY{o}{.}\PY{n}{set\PYZus{}title}\PY{p}{(}\PY{n}{name} \PY{o}{+} \PY{l+s+s2}{\PYZdq{}}\PY{l+s+s2}{: Smoothed}\PY{l+s+s2}{\PYZdq{}}\PY{p}{)}
              \PY{n}{plt}\PY{o}{.}\PY{n}{show}\PY{p}{(}\PY{p}{)}
\end{Verbatim}


    \begin{center}
    \adjustimage{max size={0.9\linewidth}{0.9\paperheight}}{output_10_0.png}
    \end{center}
    { \hspace*{\fill} \\}
    
    \begin{center}
    \adjustimage{max size={0.9\linewidth}{0.9\paperheight}}{output_10_1.png}
    \end{center}
    { \hspace*{\fill} \\}
    
    \begin{center}
    \adjustimage{max size={0.9\linewidth}{0.9\paperheight}}{output_10_2.png}
    \end{center}
    { \hspace*{\fill} \\}
    
    \begin{center}
    \adjustimage{max size={0.9\linewidth}{0.9\paperheight}}{output_10_3.png}
    \end{center}
    { \hspace*{\fill} \\}
    
    We can see the main city in the centre of the images with a mountainous
area in the top and right of the image. The white speckled bits are
cloud. Overall it is a pretty clear image, there is a darker region to
the left which is where one satellite image has been stitched together
with another.

We can see the difference in fine detail lost after the smoothing. But
that might suit us as we are not interested in the fine detail

    \subsection{Clustering}\label{clustering}

Here we define a function that runs our k-means clustering algorithm.

Clustering is an unsupervised machine learning approach. This means the
data does not contain labels, so we do not tell the algorithm what class
a particular observation (in this case a pixel in the image) should
have. Instead, the algorithm looks at the distribution of the various
features (in our case the amount of red, green and blue of each pixel)
and tells us how the data is best grouped into classes.

Clustering algorithms can be sensitive to the starting parameters, so we
should try the approach with a few different parameters.

K-Means is a pretty straightfoward approach to clustering.

In K-Means, k is the number of clusters we want it to find, which we
define beforehand. As K-Means is sensitive to starting parameters, we
will try several different values for K.

The broad steps for the algorithm are as follows

\begin{enumerate}
\def\labelenumi{\arabic{enumi}.}
\tightlist
\item
  Select the centroids for each of the K clusters - this can be done by
  randomly selecting an observation in our dataset or by defining them
  beforehand. Here we select them randomly.
\item
  For each observation, it calculates the euclidian distance in the
  feature space to the centroid of each cluster each datapoint is
  assigned to the cluster that has shortest euclidian distance. Simply
  put: it assigns each datapoint to the closest cluster.
\item
  The centroids for each cluster are recalculated based upon the mean
  (hence the name) of the features across all the observations' grouped
  in that cluster.
\item
  Repeat from step 2 until a stopping condition is met. Examples of
  these are: no observation changes cluster, the sum of the euclian
  distances of each observation and the centroid of its cluster drops
  below a threshold or a maximum number of iterations is reached.
\end{enumerate}

Below is the code to compute clusters. This uses the scikit-learn
implementation of K-Means documented here:
http://scikit-learn.org/stable/modules/generated/sklearn.cluster.KMeans.html

    \subsection{Create the functions we need to process the
data}\label{create-the-functions-we-need-to-process-the-data}

Here we create two functions. The first, cluster\_image contains the
code to run the k-means algorithm on the data. The second runs the
k\_means with multiple values for k and gets the results

    \begin{Verbatim}[commandchars=\\\{\}]
{\color{incolor}In [{\color{incolor}168}]:} \PY{k}{def} \PY{n+nf}{cluster\PYZus{}image}\PY{p}{(}\PY{n}{groups}\PY{p}{,} \PY{n}{img}\PY{p}{,} \PY{n}{method} \PY{o}{=} \PY{l+s+s2}{\PYZdq{}}\PY{l+s+s2}{random}\PY{l+s+s2}{\PYZdq{}}\PY{p}{)}\PY{p}{:}
              \PY{l+s+sd}{\PYZdq{}\PYZdq{}\PYZdq{}cluster\PYZus{}image}
          \PY{l+s+sd}{    Takes an image, represented as a numpy array and attempts to cluster the pixels}
          \PY{l+s+sd}{    in the image into a specified number of groups.}
          \PY{l+s+sd}{    By default uses random starting clusters with a specified random seed}
          \PY{l+s+sd}{    }
          \PY{l+s+sd}{    Args:}
          \PY{l+s+sd}{        groups (int): The number of groups to cluster the data into (k)}
          \PY{l+s+sd}{        img (Array): The image of dimensions (x,y,z) where z is the features to cluster over}
          \PY{l+s+sd}{        method (String): Initial starting method to use, random by default. }
          \PY{l+s+sd}{            See: http://scikit\PYZhy{}learn.org/stable/modules/generated/sklearn.cluster.KMeans.html}
          \PY{l+s+sd}{    }
          \PY{l+s+sd}{    Returns:}
          \PY{l+s+sd}{        cluster\PYZus{}labels (Array): Contains cluster labels for img in a 2\PYZhy{}D array of the same size as the first two}
          \PY{l+s+sd}{            dimensions of img}
          \PY{l+s+sd}{    }
          \PY{l+s+sd}{    \PYZdq{}\PYZdq{}\PYZdq{}}
              
              \PY{c+c1}{\PYZsh{}put into the right shape}
              \PY{n}{dims} \PY{o}{=} \PY{n}{np}\PY{o}{.}\PY{n}{shape}\PY{p}{(}\PY{n}{img}\PY{p}{)}
              \PY{n}{img\PYZus{}matrix} \PY{o}{=} \PY{n}{np}\PY{o}{.}\PY{n}{reshape}\PY{p}{(}\PY{n}{img}\PY{p}{,} \PY{p}{(}\PY{n}{dims}\PY{p}{[}\PY{l+m+mi}{0}\PY{p}{]} \PY{o}{*} \PY{n}{dims}\PY{p}{[}\PY{l+m+mi}{1}\PY{p}{]}\PY{p}{,} \PY{n}{dims}\PY{p}{[}\PY{l+m+mi}{2}\PY{p}{]}\PY{p}{)}\PY{p}{)}
              
              \PY{c+c1}{\PYZsh{}cluster}
              \PY{n}{cl} \PY{o}{=} \PY{n}{KMeans}\PY{p}{(}\PY{n}{n\PYZus{}clusters} \PY{o}{=} \PY{n}{groups}\PY{p}{,} \PY{n}{init} \PY{o}{=} \PY{n}{method}\PY{p}{)}
              \PY{n}{img\PYZus{}groups} \PY{o}{=} \PY{n}{cl}\PY{o}{.}\PY{n}{fit\PYZus{}predict}\PY{p}{(}\PY{n}{img\PYZus{}matrix}\PY{p}{)}
              
              \PY{c+c1}{\PYZsh{}create image}
              \PY{n}{cluster\PYZus{}groups} \PY{o}{=} \PY{n}{np}\PY{o}{.}\PY{n}{reshape}\PY{p}{(}\PY{n}{img\PYZus{}groups}\PY{p}{,} \PY{p}{(}\PY{n}{dims}\PY{p}{[}\PY{l+m+mi}{0}\PY{p}{]}\PY{p}{,} \PY{n}{dims}\PY{p}{[}\PY{l+m+mi}{1}\PY{p}{]}\PY{p}{)}\PY{p}{)}
              
              \PY{k}{return} \PY{n}{cluster\PYZus{}groups}
\end{Verbatim}


    \begin{Verbatim}[commandchars=\\\{\}]
{\color{incolor}In [{\color{incolor}260}]:} \PY{k}{def} \PY{n+nf}{cluster\PYZus{}ks}\PY{p}{(}\PY{n}{image}\PY{p}{,} \PY{n}{ks}\PY{p}{)}\PY{p}{:}
              
              \PY{l+s+sd}{\PYZdq{}\PYZdq{}\PYZdq{}cluster\PYZus{}ks}
          \PY{l+s+sd}{    Wrapper for cluster image. Repeats clustering for a range of values.}
          \PY{l+s+sd}{    }
          \PY{l+s+sd}{    Args:}
          \PY{l+s+sd}{        image (Array): The image of dimensions (x,y,z) where z is the features to cluster over}
          \PY{l+s+sd}{        ks (iterable): integer values of k to cluster with}
          \PY{l+s+sd}{        }
          \PY{l+s+sd}{    Returns: }
          \PY{l+s+sd}{        (dict): key:value pair where key is k clusters and value is the results in a numpy array }
          \PY{l+s+sd}{        }
          \PY{l+s+sd}{    \PYZdq{}\PYZdq{}\PYZdq{}}
              
              \PY{n}{cluster\PYZus{}labels} \PY{o}{=} \PY{p}{[}\PY{p}{]}
              
              \PY{k}{for} \PY{n}{k} \PY{o+ow}{in} \PY{n}{ks}\PY{p}{:}
                  
                  \PY{c+c1}{\PYZsh{}get cluster groups}
                  \PY{n}{group\PYZus{}labels} \PY{o}{=} \PY{n}{cluster\PYZus{}image}\PY{p}{(}\PY{n}{groups} \PY{o}{=} \PY{n}{k}\PY{p}{,} \PY{n}{img} \PY{o}{=} \PY{n}{image}\PY{p}{)}
                  \PY{n}{cluster\PYZus{}labels}\PY{o}{.}\PY{n}{append}\PY{p}{(}\PY{n}{group\PYZus{}labels}\PY{p}{)}
              
              
              \PY{n}{clusters} \PY{o}{=} \PY{p}{[}\PY{n+nb}{str}\PY{p}{(}\PY{n}{k}\PY{p}{)} \PY{k}{for} \PY{n}{k} \PY{o+ow}{in} \PY{n}{ks}\PY{p}{]}
              \PY{k}{return} \PY{n+nb}{dict}\PY{p}{(}\PY{n+nb}{zip}\PY{p}{(}\PY{n}{clusters}\PY{p}{,}\PY{n}{cluster\PYZus{}labels}\PY{p}{)}\PY{p}{)}
\end{Verbatim}


    \begin{Verbatim}[commandchars=\\\{\}]
{\color{incolor}In [{\color{incolor}266}]:} \PY{k}{def} \PY{n+nf}{plt\PYZus{}results}\PY{p}{(}\PY{n}{ks}\PY{p}{,} \PY{n}{imgs}\PY{p}{,} \PY{n}{smoothed\PYZus{}imgs}\PY{p}{,} \PY{n}{img\PYZus{}name}\PY{p}{)}\PY{p}{:}
              \PY{l+s+sd}{\PYZdq{}\PYZdq{}\PYZdq{}plt\PYZus{}results}
          \PY{l+s+sd}{    }
          \PY{l+s+sd}{    Plot results from smoothed and unsmoothed images side by side}
          \PY{l+s+sd}{    }
          \PY{l+s+sd}{    Args:}
          \PY{l+s+sd}{        ks (iterable): the value for k used to cluster}
          \PY{l+s+sd}{        img (dict): cluster results from unsmoothed image}
          \PY{l+s+sd}{        smoothed\PYZus{}img (dict): cluster results from smoothed image}
          \PY{l+s+sd}{        img\PYZus{}name (string): name of the image the results are for}
          \PY{l+s+sd}{        }
          \PY{l+s+sd}{    Returns:}
          \PY{l+s+sd}{        figs (List): the figures created from the results}
          \PY{l+s+sd}{    \PYZdq{}\PYZdq{}\PYZdq{}}
          
              \PY{n}{figs} \PY{o}{=}\PY{p}{[}\PY{p}{]}
              \PY{k}{for} \PY{n}{k} \PY{o+ow}{in} \PY{n+nb}{range}\PY{p}{(}\PY{l+m+mi}{3}\PY{p}{,}\PY{l+m+mi}{10}\PY{p}{)}\PY{p}{:}
                  \PY{n}{fig}\PY{p}{,} \PY{n}{axs} \PY{o}{=} \PY{n}{plt}\PY{o}{.}\PY{n}{subplots}\PY{p}{(}\PY{l+m+mi}{1}\PY{p}{,}\PY{l+m+mi}{2}\PY{p}{)}
                  \PY{n}{axs}\PY{p}{[}\PY{l+m+mi}{0}\PY{p}{]}\PY{o}{.}\PY{n}{imshow}\PY{p}{(}\PY{n}{imgs}\PY{p}{[}\PY{n+nb}{str}\PY{p}{(}\PY{n}{k}\PY{p}{)}\PY{p}{]}\PY{p}{)}
                  \PY{n}{axs}\PY{p}{[}\PY{l+m+mi}{0}\PY{p}{]}\PY{o}{.}\PY{n}{set\PYZus{}title}\PY{p}{(}\PY{n}{img\PYZus{}name} \PY{o}{+} \PY{l+s+s2}{\PYZdq{}}\PY{l+s+s2}{: }\PY{l+s+si}{\PYZob{}\PYZcb{}}\PY{l+s+s2}{ clusters}\PY{l+s+s2}{\PYZdq{}}\PY{o}{.}\PY{n}{format}\PY{p}{(}\PY{n}{k}\PY{p}{)}\PY{p}{)}
                  \PY{n}{axs}\PY{p}{[}\PY{l+m+mi}{1}\PY{p}{]}\PY{o}{.}\PY{n}{imshow}\PY{p}{(}\PY{n}{nat\PYZus{}smooth\PYZus{}results}\PY{p}{[}\PY{n+nb}{str}\PY{p}{(}\PY{n}{k}\PY{p}{)}\PY{p}{]}\PY{p}{)}
                  \PY{n}{axs}\PY{p}{[}\PY{l+m+mi}{1}\PY{p}{]}\PY{o}{.}\PY{n}{set\PYZus{}title}\PY{p}{(}\PY{n}{img\PYZus{}name} \PY{o}{+} \PY{l+s+s2}{\PYZdq{}}\PY{l+s+s2}{, smoothed: }\PY{l+s+si}{\PYZob{}\PYZcb{}}\PY{l+s+s2}{ clusters}\PY{l+s+s2}{\PYZdq{}}\PY{o}{.}\PY{n}{format}\PY{p}{(}\PY{n}{k}\PY{p}{)}\PY{p}{)}
                  \PY{n}{plt}\PY{o}{.}\PY{n}{show}\PY{p}{(}\PY{p}{)}
          
                  \PY{n}{figs}\PY{o}{.}\PY{n}{append}\PY{p}{(}\PY{n}{fig}\PY{p}{)}
                  
              \PY{k}{return}
\end{Verbatim}


    \subsection{Natural image}\label{natural-image}

    \begin{Verbatim}[commandchars=\\\{\}]
{\color{incolor}In [{\color{incolor}261}]:} \PY{n}{nat\PYZus{}results} \PY{o}{=} \PY{n}{cluster\PYZus{}ks}\PY{p}{(}\PY{n}{images}\PY{p}{[}\PY{l+s+s2}{\PYZdq{}}\PY{l+s+s2}{Natural}\PY{l+s+s2}{\PYZdq{}}\PY{p}{]}\PY{p}{,} \PY{n+nb}{range}\PY{p}{(}\PY{l+m+mi}{3}\PY{p}{,}\PY{l+m+mi}{10}\PY{p}{)}\PY{p}{)}
          \PY{n}{nat\PYZus{}smooth\PYZus{}results} \PY{o}{=} \PY{n}{cluster\PYZus{}ks}\PY{p}{(}\PY{n}{smooth\PYZus{}images}\PY{p}{[}\PY{l+s+s2}{\PYZdq{}}\PY{l+s+s2}{Natural}\PY{l+s+s2}{\PYZdq{}}\PY{p}{]}\PY{p}{,} \PY{n+nb}{range}\PY{p}{(}\PY{l+m+mi}{3}\PY{p}{,}\PY{l+m+mi}{10}\PY{p}{)}\PY{p}{)}
\end{Verbatim}


    \begin{Verbatim}[commandchars=\\\{\}]
{\color{incolor}In [{\color{incolor}268}]:} \PY{n}{plt\PYZus{}results}\PY{p}{(}\PY{n}{ks} \PY{o}{=} \PY{n+nb}{range}\PY{p}{(}\PY{l+m+mi}{3}\PY{p}{,}\PY{l+m+mi}{10}\PY{p}{)}\PY{p}{,} 
                      \PY{n}{imgs} \PY{o}{=} \PY{n}{nat\PYZus{}results}\PY{p}{,}
                      \PY{n}{smoothed\PYZus{}imgs} \PY{o}{=} \PY{n}{nat\PYZus{}smooth\PYZus{}results}\PY{p}{,} 
                      \PY{n}{img\PYZus{}name} \PY{o}{=} \PY{l+s+s2}{\PYZdq{}}\PY{l+s+s2}{Natural}\PY{l+s+s2}{\PYZdq{}}
                     \PY{p}{)}
\end{Verbatim}


    \begin{center}
    \adjustimage{max size={0.9\linewidth}{0.9\paperheight}}{output_19_0.png}
    \end{center}
    { \hspace*{\fill} \\}
    
    \begin{center}
    \adjustimage{max size={0.9\linewidth}{0.9\paperheight}}{output_19_1.png}
    \end{center}
    { \hspace*{\fill} \\}
    
    \begin{center}
    \adjustimage{max size={0.9\linewidth}{0.9\paperheight}}{output_19_2.png}
    \end{center}
    { \hspace*{\fill} \\}
    
    \begin{center}
    \adjustimage{max size={0.9\linewidth}{0.9\paperheight}}{output_19_3.png}
    \end{center}
    { \hspace*{\fill} \\}
    
    \begin{center}
    \adjustimage{max size={0.9\linewidth}{0.9\paperheight}}{output_19_4.png}
    \end{center}
    { \hspace*{\fill} \\}
    
    \begin{center}
    \adjustimage{max size={0.9\linewidth}{0.9\paperheight}}{output_19_5.png}
    \end{center}
    { \hspace*{\fill} \\}
    
    \begin{center}
    \adjustimage{max size={0.9\linewidth}{0.9\paperheight}}{output_19_6.png}
    \end{center}
    { \hspace*{\fill} \\}
    
    \subsubsection{Results}\label{results}

\paragraph{Images}\label{images}

The most stark difference is how much noiser our unsmoothed images look.
If we wanted to perform some kind of edge detection or segmentation, we
could do that on our smoothed images but not the unsmoothed.

As for picking out the main city. It looks like using three clusters
works ok, as do four and five. However, there are still things like
clouds and mountains that it is being grouped with,so not perfect.

Six clusters and upwards appear to be less useful and exhibit more noise
in the results - not idea for gross segmentation of land use.

We can get some metrics for the amount of land put into each group.

    \section{Agricultural}\label{agricultural}

Lets repeat the steps above with the agricultural image. This image has
already been pre-processed to identify some land use which might make
our job easier

    \begin{Verbatim}[commandchars=\\\{\}]
{\color{incolor}In [{\color{incolor} }]:} \PY{n}{agric\PYZus{}results} \PY{o}{=} \PY{n}{cluster\PYZus{}ks}\PY{p}{(}\PY{n}{images}\PY{p}{[}\PY{l+s+s2}{\PYZdq{}}\PY{l+s+s2}{Agricultural}\PY{l+s+s2}{\PYZdq{}}\PY{p}{]}\PY{p}{,} \PY{n+nb}{range}\PY{p}{(}\PY{l+m+mi}{3}\PY{p}{,}\PY{l+m+mi}{10}\PY{p}{)}\PY{p}{)}
        \PY{n}{smooth\PYZus{}agric\PYZus{}results} \PY{o}{=} \PY{n}{cluster\PYZus{}ks}\PY{p}{(}\PY{n}{smooth\PYZus{}images}\PY{p}{[}\PY{l+s+s2}{\PYZdq{}}\PY{l+s+s2}{Agricultural}\PY{l+s+s2}{\PYZdq{}}\PY{p}{]}\PY{p}{,} \PY{n+nb}{range}\PY{p}{(}\PY{l+m+mi}{3}\PY{p}{,}\PY{l+m+mi}{10}\PY{p}{)}\PY{p}{)}
\end{Verbatim}


    \begin{Verbatim}[commandchars=\\\{\}]
{\color{incolor}In [{\color{incolor}270}]:} \PY{n}{plt\PYZus{}results}\PY{p}{(}\PY{n}{ks} \PY{o}{=} \PY{n+nb}{range}\PY{p}{(}\PY{l+m+mi}{3}\PY{p}{,}\PY{l+m+mi}{10}\PY{p}{)}\PY{p}{,} 
                      \PY{n}{imgs} \PY{o}{=} \PY{n}{agric\PYZus{}results}\PY{p}{,}
                      \PY{n}{smoothed\PYZus{}imgs} \PY{o}{=} \PY{n}{smooth\PYZus{}agric\PYZus{}results}\PY{p}{,} 
                      \PY{n}{img\PYZus{}name} \PY{o}{=} \PY{l+s+s2}{\PYZdq{}}\PY{l+s+s2}{Agricultural}\PY{l+s+s2}{\PYZdq{}}
                     \PY{p}{)}
\end{Verbatim}


    \begin{center}
    \adjustimage{max size={0.9\linewidth}{0.9\paperheight}}{output_23_0.png}
    \end{center}
    { \hspace*{\fill} \\}
    
    \begin{center}
    \adjustimage{max size={0.9\linewidth}{0.9\paperheight}}{output_23_1.png}
    \end{center}
    { \hspace*{\fill} \\}
    
    \begin{center}
    \adjustimage{max size={0.9\linewidth}{0.9\paperheight}}{output_23_2.png}
    \end{center}
    { \hspace*{\fill} \\}
    
    \begin{center}
    \adjustimage{max size={0.9\linewidth}{0.9\paperheight}}{output_23_3.png}
    \end{center}
    { \hspace*{\fill} \\}
    
    \begin{center}
    \adjustimage{max size={0.9\linewidth}{0.9\paperheight}}{output_23_4.png}
    \end{center}
    { \hspace*{\fill} \\}
    
    \begin{center}
    \adjustimage{max size={0.9\linewidth}{0.9\paperheight}}{output_23_5.png}
    \end{center}
    { \hspace*{\fill} \\}
    
    \begin{center}
    \adjustimage{max size={0.9\linewidth}{0.9\paperheight}}{output_23_6.png}
    \end{center}
    { \hspace*{\fill} \\}
    
    \subsubsection{Results}\label{results}

\paragraph{Images}\label{images}

This shows similar results to the natural image. Clustering on the
smoothed image more promising.

With three groups, it looks able to seperate out the more mountainous
areas with. clouds and agricultural/urban areas. Not quite what we want.

With 4 and 5, you can pick out the urban areas of the city and
surrounding towns quite well. As you add more clusters it appears to
gradually get more noise and appears less useful.

    \subsection{Urban image}\label{urban-image}

Like the Agricultural image, this has been pre-processed except, to pick
out urban areas instead of agricutural.

    \begin{Verbatim}[commandchars=\\\{\}]
{\color{incolor}In [{\color{incolor}271}]:} \PY{n}{urban\PYZus{}results} \PY{o}{=} \PY{n}{cluster\PYZus{}ks}\PY{p}{(}\PY{n}{images}\PY{p}{[}\PY{l+s+s2}{\PYZdq{}}\PY{l+s+s2}{Urban}\PY{l+s+s2}{\PYZdq{}}\PY{p}{]}\PY{p}{,} \PY{n+nb}{range}\PY{p}{(}\PY{l+m+mi}{3}\PY{p}{,}\PY{l+m+mi}{10}\PY{p}{)}\PY{p}{)}
          \PY{n}{smooth\PYZus{}urb\PYZus{}results} \PY{o}{=} \PY{n}{cluster\PYZus{}ks}\PY{p}{(}\PY{n}{smooth\PYZus{}images}\PY{p}{[}\PY{l+s+s2}{\PYZdq{}}\PY{l+s+s2}{Urban}\PY{l+s+s2}{\PYZdq{}}\PY{p}{]}\PY{p}{,} \PY{n+nb}{range}\PY{p}{(}\PY{l+m+mi}{3}\PY{p}{,}\PY{l+m+mi}{10}\PY{p}{)}\PY{p}{)}
\end{Verbatim}


    \begin{Verbatim}[commandchars=\\\{\}]
{\color{incolor}In [{\color{incolor}273}]:} \PY{n}{plt\PYZus{}results}\PY{p}{(}\PY{n}{ks} \PY{o}{=} \PY{n+nb}{range}\PY{p}{(}\PY{l+m+mi}{3}\PY{p}{,}\PY{l+m+mi}{10}\PY{p}{)}\PY{p}{,} 
                      \PY{n}{imgs} \PY{o}{=} \PY{n}{urban\PYZus{}results}\PY{p}{,}
                      \PY{n}{smoothed\PYZus{}imgs} \PY{o}{=} \PY{n}{smooth\PYZus{}urb\PYZus{}results}\PY{p}{,} 
                      \PY{n}{img\PYZus{}name} \PY{o}{=} \PY{l+s+s2}{\PYZdq{}}\PY{l+s+s2}{Urban}\PY{l+s+s2}{\PYZdq{}}
                     \PY{p}{)}
\end{Verbatim}


    \begin{center}
    \adjustimage{max size={0.9\linewidth}{0.9\paperheight}}{output_27_0.png}
    \end{center}
    { \hspace*{\fill} \\}
    
    \begin{center}
    \adjustimage{max size={0.9\linewidth}{0.9\paperheight}}{output_27_1.png}
    \end{center}
    { \hspace*{\fill} \\}
    
    \begin{center}
    \adjustimage{max size={0.9\linewidth}{0.9\paperheight}}{output_27_2.png}
    \end{center}
    { \hspace*{\fill} \\}
    
    \begin{center}
    \adjustimage{max size={0.9\linewidth}{0.9\paperheight}}{output_27_3.png}
    \end{center}
    { \hspace*{\fill} \\}
    
    \begin{center}
    \adjustimage{max size={0.9\linewidth}{0.9\paperheight}}{output_27_4.png}
    \end{center}
    { \hspace*{\fill} \\}
    
    \begin{center}
    \adjustimage{max size={0.9\linewidth}{0.9\paperheight}}{output_27_5.png}
    \end{center}
    { \hspace*{\fill} \\}
    
    \begin{center}
    \adjustimage{max size={0.9\linewidth}{0.9\paperheight}}{output_27_6.png}
    \end{center}
    { \hspace*{\fill} \\}
    
    \subsubsection{Results}\label{results}

\subsubsection{images}\label{images}

As with the previous two images, four and five clusters show differences
between the surrounding grassland and the city in the centre, but
struggle to sepearate the city from the clouds while adding more
clusters just adds more noise to the result.

    \subsection{Vegetation image}\label{vegetation-image}

This image uses parts of the infared spectrum to show areas of
vegetation so may be able to seperate out the urban areas from the city.

    \begin{Verbatim}[commandchars=\\\{\}]
{\color{incolor}In [{\color{incolor}275}]:} \PY{n}{veg\PYZus{}results} \PY{o}{=} \PY{n}{cluster\PYZus{}ks}\PY{p}{(}\PY{n}{images}\PY{p}{[}\PY{l+s+s2}{\PYZdq{}}\PY{l+s+s2}{Vegetation}\PY{l+s+s2}{\PYZdq{}}\PY{p}{]}\PY{p}{,} \PY{n+nb}{range}\PY{p}{(}\PY{l+m+mi}{3}\PY{p}{,}\PY{l+m+mi}{10}\PY{p}{)}\PY{p}{)}
          \PY{n}{smooth\PYZus{}veg\PYZus{}results} \PY{o}{=} \PY{n}{cluster\PYZus{}ks}\PY{p}{(}\PY{n}{smooth\PYZus{}images}\PY{p}{[}\PY{l+s+s2}{\PYZdq{}}\PY{l+s+s2}{Vegetation}\PY{l+s+s2}{\PYZdq{}}\PY{p}{]}\PY{p}{,} \PY{n+nb}{range}\PY{p}{(}\PY{l+m+mi}{3}\PY{p}{,}\PY{l+m+mi}{10}\PY{p}{)}\PY{p}{)}
\end{Verbatim}


    \begin{Verbatim}[commandchars=\\\{\}]
{\color{incolor}In [{\color{incolor}277}]:} \PY{n}{plt\PYZus{}results}\PY{p}{(}\PY{n}{ks} \PY{o}{=} \PY{n+nb}{range}\PY{p}{(}\PY{l+m+mi}{3}\PY{p}{,}\PY{l+m+mi}{10}\PY{p}{)}\PY{p}{,} 
                      \PY{n}{imgs} \PY{o}{=} \PY{n}{veg\PYZus{}results}\PY{p}{,}
                      \PY{n}{smoothed\PYZus{}imgs} \PY{o}{=} \PY{n}{smooth\PYZus{}veg\PYZus{}results}\PY{p}{,} 
                      \PY{n}{img\PYZus{}name} \PY{o}{=} \PY{l+s+s2}{\PYZdq{}}\PY{l+s+s2}{Vegetation}\PY{l+s+s2}{\PYZdq{}}
                     \PY{p}{)}
\end{Verbatim}


    \begin{center}
    \adjustimage{max size={0.9\linewidth}{0.9\paperheight}}{output_31_0.png}
    \end{center}
    { \hspace*{\fill} \\}
    
    \begin{center}
    \adjustimage{max size={0.9\linewidth}{0.9\paperheight}}{output_31_1.png}
    \end{center}
    { \hspace*{\fill} \\}
    
    \begin{center}
    \adjustimage{max size={0.9\linewidth}{0.9\paperheight}}{output_31_2.png}
    \end{center}
    { \hspace*{\fill} \\}
    
    \begin{center}
    \adjustimage{max size={0.9\linewidth}{0.9\paperheight}}{output_31_3.png}
    \end{center}
    { \hspace*{\fill} \\}
    
    \begin{center}
    \adjustimage{max size={0.9\linewidth}{0.9\paperheight}}{output_31_4.png}
    \end{center}
    { \hspace*{\fill} \\}
    
    \begin{center}
    \adjustimage{max size={0.9\linewidth}{0.9\paperheight}}{output_31_5.png}
    \end{center}
    { \hspace*{\fill} \\}
    
    \begin{center}
    \adjustimage{max size={0.9\linewidth}{0.9\paperheight}}{output_31_6.png}
    \end{center}
    { \hspace*{\fill} \\}
    
    \subsubsection{Results}\label{results}

\paragraph{images}\label{images}

Otherwise it is simlar to before with more clusters after 6, just adding
noise.

    \subsection{Combining the images}\label{combining-the-images}

In machine learning, more data is generally better \emph{if} the data is
relevant to the problem you are trying to solve. Since all the images
appear to go some way to helping us, it makes sense to combine the data
and repeat the k-means.

    \begin{Verbatim}[commandchars=\\\{\}]
{\color{incolor}In [{\color{incolor}217}]:} \PY{n}{all\PYZus{}img} \PY{o}{=} \PY{n}{np}\PY{o}{.}\PY{n}{concatenate}\PY{p}{(}\PY{p}{[}\PY{n}{images}\PY{p}{[}\PY{l+s+s2}{\PYZdq{}}\PY{l+s+s2}{Natural}\PY{l+s+s2}{\PYZdq{}}\PY{p}{]}\PY{p}{,} 
                                    \PY{n}{images}\PY{p}{[}\PY{l+s+s2}{\PYZdq{}}\PY{l+s+s2}{Vegetation}\PY{l+s+s2}{\PYZdq{}}\PY{p}{]}\PY{p}{,}
                                    \PY{n}{images}\PY{p}{[}\PY{l+s+s2}{\PYZdq{}}\PY{l+s+s2}{Urban}\PY{l+s+s2}{\PYZdq{}}\PY{p}{]}\PY{p}{,}
                                    \PY{n}{images}\PY{p}{[}\PY{l+s+s2}{\PYZdq{}}\PY{l+s+s2}{Agricultural}\PY{l+s+s2}{\PYZdq{}}\PY{p}{]}
                                   \PY{p}{]}\PY{p}{,}
                                   \PY{n}{axis} \PY{o}{=} \PY{l+m+mi}{2}
                                  \PY{p}{)}
          
          \PY{n}{images}\PY{p}{[}\PY{l+s+s2}{\PYZdq{}}\PY{l+s+s2}{Combined}\PY{l+s+s2}{\PYZdq{}}\PY{p}{]} \PY{o}{=} \PY{n}{all\PYZus{}img}
          
          \PY{n}{all\PYZus{}smooth\PYZus{}img} \PY{o}{=} \PY{n}{np}\PY{o}{.}\PY{n}{concatenate}\PY{p}{(}\PY{p}{[}\PY{n}{smooth\PYZus{}images}\PY{p}{[}\PY{l+s+s2}{\PYZdq{}}\PY{l+s+s2}{Natural}\PY{l+s+s2}{\PYZdq{}}\PY{p}{]}\PY{p}{,} 
                                           \PY{n}{smooth\PYZus{}images}\PY{p}{[}\PY{l+s+s2}{\PYZdq{}}\PY{l+s+s2}{Vegetation}\PY{l+s+s2}{\PYZdq{}}\PY{p}{]}\PY{p}{,}
                                           \PY{n}{smooth\PYZus{}images}\PY{p}{[}\PY{l+s+s2}{\PYZdq{}}\PY{l+s+s2}{Urban}\PY{l+s+s2}{\PYZdq{}}\PY{p}{]}\PY{p}{,}
                                           \PY{n}{smooth\PYZus{}images}\PY{p}{[}\PY{l+s+s2}{\PYZdq{}}\PY{l+s+s2}{Agricultural}\PY{l+s+s2}{\PYZdq{}}\PY{p}{]}
                                          \PY{p}{]}\PY{p}{,}
                                          \PY{n}{axis} \PY{o}{=} \PY{l+m+mi}{2}
                                         \PY{p}{)}
          
          \PY{n}{smooth\PYZus{}images}\PY{p}{[}\PY{l+s+s2}{\PYZdq{}}\PY{l+s+s2}{Combined}\PY{l+s+s2}{\PYZdq{}}\PY{p}{]} \PY{o}{=} \PY{n}{all\PYZus{}smooth\PYZus{}img}
\end{Verbatim}


    \begin{Verbatim}[commandchars=\\\{\}]
{\color{incolor}In [{\color{incolor}279}]:} \PY{n}{combo\PYZus{}results} \PY{o}{=} \PY{n}{cluster\PYZus{}ks}\PY{p}{(}\PY{n}{images}\PY{p}{[}\PY{l+s+s2}{\PYZdq{}}\PY{l+s+s2}{Combined}\PY{l+s+s2}{\PYZdq{}}\PY{p}{]}\PY{p}{,} \PY{n+nb}{range}\PY{p}{(}\PY{l+m+mi}{3}\PY{p}{,}\PY{l+m+mi}{10}\PY{p}{)}\PY{p}{)}
          \PY{n}{smooth\PYZus{}combo\PYZus{}results} \PY{o}{=} \PY{n}{cluster\PYZus{}ks}\PY{p}{(}\PY{n}{smooth\PYZus{}images}\PY{p}{[}\PY{l+s+s2}{\PYZdq{}}\PY{l+s+s2}{Combined}\PY{l+s+s2}{\PYZdq{}}\PY{p}{]}\PY{p}{,} \PY{n+nb}{range}\PY{p}{(}\PY{l+m+mi}{3}\PY{p}{,}\PY{l+m+mi}{10}\PY{p}{)}\PY{p}{)}
\end{Verbatim}


    \begin{Verbatim}[commandchars=\\\{\}]
{\color{incolor}In [{\color{incolor}281}]:} \PY{n}{plt\PYZus{}results}\PY{p}{(}\PY{n}{ks} \PY{o}{=} \PY{n+nb}{range}\PY{p}{(}\PY{l+m+mi}{3}\PY{p}{,}\PY{l+m+mi}{10}\PY{p}{)}\PY{p}{,} 
                      \PY{n}{imgs} \PY{o}{=} \PY{n}{combo\PYZus{}results}\PY{p}{,}
                      \PY{n}{smoothed\PYZus{}imgs} \PY{o}{=} \PY{n}{smooth\PYZus{}combo\PYZus{}results}\PY{p}{,} 
                      \PY{n}{img\PYZus{}name} \PY{o}{=} \PY{l+s+s2}{\PYZdq{}}\PY{l+s+s2}{Combined}\PY{l+s+s2}{\PYZdq{}}
                     \PY{p}{)}
\end{Verbatim}


    \begin{center}
    \adjustimage{max size={0.9\linewidth}{0.9\paperheight}}{output_36_0.png}
    \end{center}
    { \hspace*{\fill} \\}
    
    \begin{center}
    \adjustimage{max size={0.9\linewidth}{0.9\paperheight}}{output_36_1.png}
    \end{center}
    { \hspace*{\fill} \\}
    
    \begin{center}
    \adjustimage{max size={0.9\linewidth}{0.9\paperheight}}{output_36_2.png}
    \end{center}
    { \hspace*{\fill} \\}
    
    \begin{center}
    \adjustimage{max size={0.9\linewidth}{0.9\paperheight}}{output_36_3.png}
    \end{center}
    { \hspace*{\fill} \\}
    
    \begin{center}
    \adjustimage{max size={0.9\linewidth}{0.9\paperheight}}{output_36_4.png}
    \end{center}
    { \hspace*{\fill} \\}
    
    \begin{center}
    \adjustimage{max size={0.9\linewidth}{0.9\paperheight}}{output_36_5.png}
    \end{center}
    { \hspace*{\fill} \\}
    
    \begin{center}
    \adjustimage{max size={0.9\linewidth}{0.9\paperheight}}{output_36_6.png}
    \end{center}
    { \hspace*{\fill} \\}
    
    \subsubsection{Results}\label{results}

\paragraph{images}\label{images}

It is a similar story for our combined image dataset. Four and five.

    \subsection{Overall}\label{overall}

\begin{itemize}
\tightlist
\item
  Smoothed images appear better for gross segmentation
\item
  Clouds get grouped in with terrian, ideally any images will need to be
  free of clouds.
\item
  Unsupervised methods may be of some use. If images can come with gis
  data, it may be possible to use this information as labels in a
  superivsed machine learning approach which could be more useable.
\end{itemize}


    % Add a bibliography block to the postdoc
    
    
    
    \end{document}
